
\section{Vorlesung}
\subsection{Was ist Statistik}

  \begin{tabular}{| c | c | c | c |}
    \hline
      \thead{Deskriptiv} & \thead{Inferenz} & \thead{amtliche Statistik} & \thead{Explorative Statistik} \\
    \hline
      \makecell{Merkmale, Zusammenhänge \\Grafische Darstellung \\Lage und Streumaße} &
      \makecell{GG  $\leftrightarrow$ Stichprobe\\ Stichprobenfehler(sample error)}   &
      \makecell{von Institutionen \\in Auftrag gegeben} &
      \makecell{Zusammenhänge in \\Daten finden, "Big Data",\\ etabliert in Wirtschaft} \\
    \hline
  \end{tabular}


\subsection{Grundgesamtheit und Stichprobe}
Grundgesamtheit: Menge der Objekte für die die Aussage der Untersuchung gelten soll\\
Stichprobe: regelgeleitete Auswahl einer Teilmenge von Elementen aus der Grundgesamtheit\\
Stichprobenfehler/Sampling Error: Merkmalsausprägung ist in GG und in Stichprobe unterschiedlich\\
Kleer hatte noch sampling:

Flick Stichproben Kapitel 4



\subsection{Variablen}
\subsubsection{Wiederholung Empirie 1 Hussy-Schreier-Echterhoff:}

\underline{1. Beschreiben:} Variable B hängt mit Variabale A zusammen\hspace{1cm}A \textemdash{} B\\\\
\underline{2. Erklären:}~~~~~~~~ Variable B ist abhängig von Variable A\hspace{2cm}A\textrightarrow B\\
Zusammenhang:\\
positiver Zusammenhang: A$\uparrow$ B$\uparrow$\\
negativer Zusammenhang: A$\uparrow$ B$\downarrow$ oder A$\downarrow$ B$\uparrow$\\ Kein Zusammenhang\\
\\ Kausalrelation:\\
A$\longrightarrow$B\\
A$\longleftarrow$B\\
A$\longleftrightarrow$B\\
\\Beide Zusammen:\\
A$\uparrow$ $\longrightarrow$ B$\downarrow$\\\\
\underline{3. erste und zweite Ordnung}\\
1. Ordnung: A $\longrightarrow$ B\\
2. Ordnung: A \textrightarrow x \textrightarrow B\\
x ist intervenierende Variable

\subsection{Skalenniveaus und diskret/stetig}


\begin{adjustbox}{valign=t}
\begin{forest}
 [Kategorial[Nominal[ungeordnet\\(Haarfarbe)]][Ordinal[geordnet\\(Schulnoten)]]]
\end{forest}
\end{adjustbox}\qquad
\begin{adjustbox}{valign=t}
\begin{forest}
 [Metrisch  [Intervall[Konstante\\Abstände\\(Temperatur \celsius)]][Ratio[natürlicher\\Nullpunkt\\(Kelvin)]]]
\end{forest}
\end{adjustbox}
\vspace{0.5cm}\\
Pseudometrisch: Ordinal wird oft als Intervall behandelt, wenn genug Ausprägungen vorhanden sind\\
Welches Skalenniveau haben folgende Variablen?
\begin{enumerate}
\item Selbstvertrauen: hoch, mittel, niedrig \solution{Ordinal}
\item Jahreszahl (z.B. 1982)                 \solution{Intervall}
\item Alter                                  \solution{Ratio}
\item Geschlecht                             \solution{Nominal}
\item Anzahl erreichter Creditpoints         \solution{Ratio, 1CP=30h\textrightarrow so als ob nach <Stunden> gefragt wird}
\item Einkommen                              \solution{Ratio}

\end{enumerate}
\vspace{0.5cm}
diskret: Endlich Viele Werte können angenommen werden~1--------------------------2
\\stetig:~~~~Annahme eines beliebigen Wertes in Intervall~~~~~~~~~~~1--1.1--1.127587-------2
\vspace{0.5cm}
\\Sind die folgenden Variablen diskret oder stetig?
\begin{enumerate}
\item Gewicht                                \solution{Stetig}
\item Jahreszahl (z.B. 1982)                 \solution{Diskret}
\item Alter                                  \solution{Stetig}
\item Exakte Zeit eines 100m Läufers         \solution{Stetig}
\item Geschlecht                             \solution{Diskret}
\item Einkommen                              \solution{Stetig}
\end{enumerate}
\vspace{0.5cm}
\underline{Kleer hatte noch:}
\\dichotom: 2 Ausprägungen
\\polytom: Mehr als 2 Ausprägungen
\\
\\Makro: aggregierte Daten
\\Mikro: Individualdaten
\\
\\latent: nicht direkt beobachtbar
\\manifest: messbar
\\
