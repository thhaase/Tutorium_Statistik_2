\section{Vorlesung}

\subsection{Kreuztabelle/ Kontingenztafel}

  \begin{itemize}
    \item für nominale/ordinale Variablen
    \item Konvention: Zeile-Abhängig/Spalte-unabhängig
  \end{itemize}
\vspace{1cm}
Beispieltabelle(fiktional): \\
 \begin{itemize}
   \item [x] =~~~Studierende besuchen das Tutorium
   \item [y] =~~~Studierende bestehen die Statistikklausur
   \item [] Nichtantritt = nicht bestanden
 \end{itemize}

 \begin{tabular}{p{2cm}|p{2cm}|p{2cm}|p{2cm}}
    {}              & Tutorium besucht & Tutorium nicht besucht & \textit{Gesamt}
\\\hline
    bestanden       & 9                & 59                     & 68
\\\hline
    nicht bestanden & 2                & 14                     & 16
\\\hline
    \textit{Gesamt} & 11               & 73                     & 84
 \end{tabular}

 \subsection{Übung}
  \begin{enumerate}
    \item Wie hoch ist der relative Anteil aller eingeschriebenen Studierenden die das Tutorium nicht besucht und die Klausur nicht bestanden haben?
      \solution{$\frac{14}{84} = 16,6\%$}

    \item Wie hoch ist der relative Anteil aller eingeschriebenen Studierenden die die Klausur bestanden haben?
      \solution{$\frac{68}{84} = 80,9\%$}

    \item Wie hoch ist der relative Anteil der Tutoriumsbesucher, die die Klausur bestanden haben?
      \solution{$\frac{9}{11} = 81,8\%$}

    \item Wie viel Prozent aller bestehenden Studierenden haben das Tutorium besucht?
      \solution{$\frac{9}{68} = 13,2\%$}
 \end{enumerate}
\vspace{1cm}
Ergänze folgende Kreuztabelle (\href{https://www.destatis.de/DE/Themen/Gesellschaft-Umwelt/Gesundheit/Gesundheitszustand-Relevantes-Verhalten/Tabellen/liste-rauchverhalten.html#119170}{Statistisches Bundesamt 2021}):\\
\begin{tabular}{p{2cm}|p{2cm}|p{2cm}|p{2cm}}
   Geschlecht x Rauchen & Raucher            & Nichtraucher           & \textit{Gesamt}
\\\hline
   männlich             & 5059               & A                      & 22 684
\\\hline
   weiblich             & B                  & C                      & 23 547
\\\hline
   \textit{Gesamt}      & 8 738              & D                      & 46 231
\end{tabular}

\begin{enumerate}
  \item A \solution{22684 - 5059 = 17625}
  \item B \solution{8738  - 5059 = 3679 }
  \item C \solution{23547 - 3679 = 19868}
  \item D \solution{17625 + 19868= 37493}
\end{enumerate}
