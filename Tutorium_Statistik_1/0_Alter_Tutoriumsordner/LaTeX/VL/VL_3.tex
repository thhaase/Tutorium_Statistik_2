\section{Vorlesung}


\subsection{Lagemaße/Maße der zentralen Tendenz}
\begin{tabular}{r|l|l}
     Modus                  & Wert kommt am häufigsten vor       & ratio, intervall, ordinal, nominal\\
     Median                 & Teilt Menge in 2 gleichgroße Teile & ratio, intervall, ordinal \\
     arithmetisches Mittel  & Durchschnitt                       & ratio, intervall
\end{tabular}


\subsubsection{Median}
\begin{tabular}{r|c|l}
     n - ungerade &$\widetilde{x}~=~x_{\frac{n+1}{2}}$&  \\
     {}&{}&{}\\
     n - gerade   &$\widetilde{x}~=~\frac{x_{\frac{n}{2}}+x_{\frac{n}{2}+1}}{2}$&
\end{tabular}\\\\
\\ungerade:
\begin{tabular}{cc|c|cc}
     $x_{1}$&$x_{2}$&$x_{3}$&$x_{4}$&$x_{5}$  \\
     3&5&6&8&12
\end{tabular}\\
mittlerer Wert = 6\\
\\
gerade :
\begin{tabular}{cc|cc|cc}
     $x_{1}$&$x_{2}$&$x_{3}$&$x_{4}$&$x_{5}$&$x_{6}$  \\
     3&5&6&8&12&13
\end{tabular}\\
Durchschnitt der mittleren beiden Werte = 7


\subsubsection{arithmetische Mittel}
\begin{equation*}
    \overline{x}=\frac{\sum_{i=1}^{n}x_{i}}{n}~~~~~\leftarrow \textrm{Datenmatrix (``normale'' Formel)}
\end{equation*}
  \vspace{0.5cm}
\begin{equation*}
     \overline{x}=\frac{\sum_{k=1}^{m}(x_{k}\cdot f_{{x}_{k}})}{n}~~~~~\leftarrow \textrm{Häufigkeitstabelle (``spezial'' Formel)}
\end{equation*}


\underline{Erläuterung der Gleichungen:}\\
Um die mittlere Antwort, einen ``Durchschnitt'', zu berechnen werden zuerst alle Antworten die gegeben wurden aufsummiert und im zweiten Schritt durch die Anzahl der Antworten (n) geteilt.\\

In der Datenmatrix/Urliste sind alle Antworten als $x_i$'s direkt ablesbar. Diese können einfach aufsummiert werden. Die Anzahl aller Antworten (n) kann an der ID des letzten Falls abgelesen werden (sofern keine Fälle dazwischen herausgefiltert wurden).\\

In der Häufigkeitstabelle kann man die einzelnen Antworten nicht so direkt ablesen wie in der Datenmatrix. Jedoch wissen wir, dass bspw. 400 Personen Antwortausprägung 1 gegeben haben, 600 Antwortausprägung 2 usw.. Antwort 1 kommt also 500 \textit{mal} in der Datenmatrix vor, Antwort 2 600 \textit{mal} usw.. Wir rechnen also die jeweilige Antwort \textit{MAL} die Anzahl wie oft diese Antwort angegeben wurde. Die Anzahl aller Antworten (n) wird ermittelt indem die Häufigkeiten der einzelnen Antwortausprägungen addiert werden.

\subsection{Dispersionsmaße/Lagemaße und Verteilungsformen}
\textbf{Verteilungsübersicht:}\\
\begin{tabular}{|c|c|}
\hline
    Modus < Median < arithmet.Mittel & linkssteil/rechtsschief
\\ \hline
    arithmet.Mittel < Median < Modus & rechtssteil/linksschief
\\ \hline
    2 Modi, Median = arithmet.Mittel, Modus weicht stark ab & bimodal
\\ \hline
    arithmet.Mittel,Modalwert und Median fast gleich & symetrisch\\
  \hline
\end{tabular}


\subsubsection{Übung}
\begin{enumerate}
\item Berechne den Median für folgende Werte: 5,2,4,4,3,5,8 \solution{4}
\item Was ist der Modus der folgenden Werte? 3,3,4,4,4,5,5,5,5,2,2,1,1,0 \solution{5}
\item Um welche Art der Verteilung besitzt folgende Werte: arithmetisches Mittel =  24, Modus = 32, Median = 27 \solution{rechtssteil/linksschief}
\end{enumerate}
